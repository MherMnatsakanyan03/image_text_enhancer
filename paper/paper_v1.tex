\documentclass[sigconf]{acmart}
\usepackage{booktabs}
\usepackage{algorithm}
\usepackage{url}


%% \BibTeX command to typeset BibTeX logo in the docs
\AtBeginDocument{%
  \providecommand\BibTeX{{%
    \normalfont B\kern-0.5em{\scshape i\kern-0.25em b}\kern-0.8em\TeX}}}

%% These commands are for a PROCEEDINGS abstract or paper.
\settopmatter{printacmref=true} % Removes citation information below abstract
%% Was false before, we need to check why its not working as intended
\renewcommand\footnotetextcopyrightpermission[1]{} % removes footnote with conference information in 

\acmConference[AEPRO 2026]{AEPRO 2026: Algorithm Engineering Projects}{March 1}{Jena, Germany}

% convert text to title case
% http://individed.com/code/to-title-case/

% that helps you to formulate your sentences
% https://www.deepl.com/translator

\begin{document}

%%
%% The "title" command has an optional parameter,
%% allowing the author to define a "short title" to be used in page headers.
%% TODO: think about good title
\title{Image Enhancement: A Lightweight, Easy-to-Use, and Efficient C++ Library}
\subtitle{\large Algorithm Engineering 2026 Project}

%%
%% The "author" command and its associated commands are used to define
%% the authors and their affiliations.

\author{Daniel Motz}
\affiliation{%
  \institution{Friedrich Schiller University}
  \city{Jena}
  \country{Germany}}
\email{daniel.motz@uni-jena.de}

\author{Leonard Teschner}
\affiliation{%
  \institution{Friedrich Schiller University}
  \city{Jena}
  \country{Germany}}
\email{leonard.teschner@uni-jena.de}

\author{Mher Mnatsakanyan}
\affiliation{%
  \institution{Friedrich Schiller University}
  \city{Jena}
  \country{Germany}}
\email{mher.mnatsakanyan@uni-jena.de}

%% The abstract is a short summary of the work to be presented in the article.
\begin{abstract}
  TBC
\end{abstract}

%%
%% Keywords. The author(s) should pick words that accurately describe
%% the work being presented. Separate the keywords with commas.
\keywords{task parallelism, WebAssembly, OpenMP, image processing, C++}

%%
%% This command processes the author and affiliation and title
%% information and builds the first part of the formatted document.
\maketitle

\section{Introduction}
\label{sec:intro}

Possible Research Questions
\begin{enumerate}
  \item Easily extensible Open Source Pipeline, ready to use out of the box with default options giving a perfectly good image, completely written in C++ and licensed under MIT-0 and CeCILL-C, which are both commercially available (compared to e.g. unpaper) and has no big dependencies (unlike unpaper, which uses ffmpeg) and only uses CImg, which is portable (uses OpenMP), simple, self-contained, and lightweight.
\end{enumerate}


\subsection{Background}
\label{sec:background}

The execution of performance-critical applications in web browsers has changed with the introduction of WebAssembly (WASM). Although WASM offers near-native execution performance, utilising multi-core architectures with parallel programming is still a relatively new field. While native C++ programs use OpenMP to easily implement parallelism, support for OpenMP in WASM remains uncertain. 

Was ist WASM?

Was machen wir mit unserer Pipeline?

\begin{enumerate}
  \item easy to use
  \item out of the box good results
  \item low dependencies (only CImg (doesnt have other dependencies) and OpenMP )
  \item only image enhancement, no OCR
  \item MIT-0 or CeCILL-C license (commercially usable)
  \item modular (users can choose which steps to apply)
\end{enumerate}

In this paper, we examine whether and how well OpenMP-based parallelism is compatible with WebAssembly by using a lightweight image enhancement pipeline implemented in C++ with OpenMP. First, we demonstrate that the pipeline can be compiled for WASM, then we evaluate its performance in native execution and in WebAssembly. We then compare the results.

\subsection{Related Work}
\label{sec:related-work}
\begin{comment}
  
Since the introduction of WASM in 2017 \cite{WASM_Haas_2017}, research has progressed in a variety of ways. For instance, significant investment has gone into the 'internal' WASM runtime, either to add features or to investigate its performance and energy efficiency . Another area of research has investigated the use of WASM in various application \cite{wasm_runtimes_survey_zhang_2025} areas, such as web applications, the Internet of Things, and high-performance computing \cite{wasm_runtimes_survey_zhang_2025}. 
Of particular mention is the work of Chadha et al., who examine the use of WASM in the HPC domain in \cite{Mpi_Wasm_Chadha_2023}. Since parallelisation is essential for HPC applications, they have developed a WASM embedder that enables the MPI library to be used in WASM applications.  
\end{comment}

\begin{comment}
  Tools die Document Enhancement bereits können:
  - OpenCV -> viel zu groß und abhöngig von vielen anderen Bibliotheken, BSD Lizenz
  - Tesseract - Open Source OCR Engine, die einige Bildverbesserungsfunktionen bietet, aber hauptsächlich auf OCR ausgerichtet ist, nicht einfach erweiterbar, preprocessing sind nicht als allgemeine Enhnancement Methoden gedacht, Apache lizenz
  - Leptonica - C-Bibliothek für Bildverarbeitung, hauptsächlich auf OCR ausgerichtet, BCD Lizenz
  - ImageMagick - Umfassende Kommandozeilen-Tool und Bibliothek für Bildbearbeitung, aber nicht speziell auf die Verbesserung von gescannten oder fotografierten Textbildern ausgerichtet, viele Abhängigkeiten, nicht so einfach erweiterbar, Apache Lizenz
  - Unpaper - spezialisiert auf Nachbearbeitung von Scans von Büchern, viele Abhängigkeiten (ffmpeg etc), nicht so einfach erweiterbar und stark spezialisiert, GPL Lizenz
\end{comment} 

Image Enhancement und Dokumenten Preprocessing wurden von einer weitreichenden Anzahl an open-source Tools und kommerzieller Software abgedeckt. Sie reichen von umfassenden Bildbearbeitungsbibliotheken, Object Character Recognition OCR Bibliotheken, bis hin zu spezialisierten Tools für die Nachbearbeitung von gescannten Dokumenten. 


\label{sec:general-purpose-libraries}
Eine der weitverbreitesten Bibliothekt für allgemeine Bildverarbeitung ist OpenCV. OpenCV wurde von Bradski entwickelt und bietet eine Vielzahl von Funktionen und Algorithmen für die Bildverarbeitung, beispielsweise Filter, Transformationen, Segmentierunt und Feature-Erkennung, sowie Machine-und Deep-learning. OpenCV ist in C++ geschrieben und bietet eine modulare, Plattformübergreifende Architektur, die es einfach macht, es in verschiedenen Anwendungen zu integieren. Allerdings ist OpenCV eine sehr umfassende Bibliothek, die viele Funktionen und Abhängigkeiten enthählt, was zu einem Overhead führt, wenn nur ein bruchteil der Funktionen genutzt wird. Zudem ist OpenCV unter der BSD-Lizenz lizenziert. 
Ähnlich zu OpenCV bietet ImageMagick eine umfassende suite für die Bildbearbeitung, als Kommandozeilen-Tool und als Bibliothek. Es bietet Funktionen wie Formatikonvertierung, Größenänderung, Filterung und Farbkorrektur. Es ist jedoch nicht speziell auf die Verbesserung von gescannten oder fotografierten Bildern ausgerichtet, sondern eher auf allgemeine Bildbearbeitung. Die Vielzahl an Funktionen und Möglichkeiten resultieren in einer vielzahl an Abhängigkeiten, und einer großen Coedbase. Es ist unter der Apache-Lizenz lizenziert.

\label{sec:ocr-libraries}
Bei der Dokumenten-Analyse wird häufig die open-source OCR engine Tesseract verwendet. Tesseract wurde von Smith in 2007 beschrieben und hat sich in eine der führenden OCR-Engines entwickelt. Es bietet neben der Texterkennung auch einige Funktionen zur Bildverbesserung, dessen primärer Fokus ist jedoch in der OCR liegt. Die Bildverbesserungsfunktionen sind typischerweise in den OCR-Workflow integriert und stehen nicht als allgemeine Bildverbesserungsfunktionen zur verfügung. Tesseract ist unter der Apache-Lizenz lizenziert.
Eine ähnliche Bibliothek, die sich auf OCR spezialisiert hat, ist die C basierte Leptonica Bibliothek. Leptonica bietet Funktionen für die Bildverarbeitung, die häufig im backend von OCR-Engines wie Tesseract verwendet werden. Es bietet Funktionen wie Binarisierung, Rauschunterdrückung und Morphologische Operationen, die für die Vorbereitung von Bildern für die OCR-Erkennung wichtig sind. Obwohl Leptonica relativ lightweight im verlgeich zu den anderen Bibliotheken ist, dient sie hauptsächlich als low-level Bibliothek und bietet beispielsweise keine ready-to-use konfigurierbare verarbeitungs Pipelines an. Außerdem ist sie in C geschrieben, was die Integration in C++ Anwendungen erschwert. Leptonica ist unter der BSD-Lizenz lizenziert.

\label{sec:specialized-post-processing}
Eine mehr spezialisiertes Tool ist unpaper, das sich auf die Nachbearbeitung von gescannten Dokumenten spezialisiert hat. Es bietet Funktionen wie Deskewing, Binarisierung und Rauschunterdrückung, die speziell für die Verbesserung von gescannten Dokumenten entwickelt wurden. Unpaper ist jedoch stark spezialisiert und bietet nicht die Flexibilität, um es in custom C++ Anwendungen zu integireren. Basierend auf der Konfiguration benötigt unpaper viele Abhängigkeiten, was die Nutzung erschwert. Unpaper ist unter der GPL-Lizenz lizenziert, was die kommerzielle Nutzung einschränkt.

\subsection{Our Contribution}
\label{sec:our-contribution}

Im Kontrast zu den beschriebenen, existierenden Tools und Bibliotheken, ist unser Ansatz darauf ausgelegt, minimalistisch, erweiterbar und einfach zu nutzen zu sein. Anstatt eine umfassende Bildverarbeitungsbibliothek oder eine OCR-Engine zu sein, konzentrieren wir uns auf eine lightweight, modulare Image Enhancement Pipeline, welche ausschließlich in C++ geschrieben ist und nur eine Abhängigkeit (CImg) hat, die selbst leichtgewichtig und portable ist. Durch die Verwendung von multithreading mit OpenMP können wir die Pipeline effizient ausführen und so auch große Bilder schnell verarbeiten. 
Unsere Arbeit unterscheidet sich von OpenCV und ImageMagick, durch die bewusste Einschränkung auf die Bildverbesserung und durch die kleingehlatene Architektur. Im Gegensatz zu Tesseract und Leptonica, ist unsere Pipeline nicht in einen OCR-Workflow integriert, und ist nicht durch low-level Primitive Funktionen beschränkt, sondern bietet eine konfigurierbare, standalone, ready-to-use Pipeline. Unsere Pipeline ist nicht ausschließlich auf die Nachbearbeitung von gescannten Dokumenten spezialisiert, sondern ist als allgemeine Image Enhancement Pipeline konzipiert, die für eine Vielzahl von Anwendungen genutzt werden kann.
Zusätzlich sichert unser Lizenzmodell (MIT-0 oder CeCILL-C) die Nutzung in der Forschung sowie in kommerziellen Anwendungen. Durch die kombinierung von minimalistischen Abhängigkeiten, eienr modularen Achrichtektur, portabilität durch C++ und OpenMP, sowie einer kommerziell nutzbaren Lizenz, füllt unsere Pipeline eine Lücke in zwischen large-scale general-purpose Bildverarbeitungsbibliotheken und hochspezialisierten Tools für die Dokumentenverbesserung.

\begin{comment}
  Uns ist die effiziente Verarbeitung in der Pipeline wichtig. Daher vergleichen wir uns mit traditionellen open-source Binarisierungsmethoden und Bildverbesserungsverfahren und lassen bewusst Deep Learning Ansätze, sowie zahlungspflichtige Software außen vor.
\end{comment}

The pipeline is provided both as a executable and as a C++ library containing the individual methods. To efficiently process large amounts of image data, the implementation uses parallelization via OpenMP and other optimization methods such as loop blocking. The CImg library \cite{cimg} is used for the basic image processing methods.

\subsection{Outline}
\label{sec:outline}

\begin{comment}
  welche experimente werden wir durchführen?
\end{comment}

This paper is structured as follows: Section \ref{sec:pipeline} provides a detailed description of the developed pipeline and its methods. Section \ref{sec:experiments} demonstrates the performance of our pipeline using experiments. Finally, Section \ref{sec:conclusions} summarises our results and provides an outlook on possible future work.

\section{The Pipeline}
\label{sec:pipeline}

\section{Pipeline Optimizations}
\label{sec:optimazation}

\section{Experiments}
\label{sec:experiments}

\begin{comment}
  In der Literatur wird die Binarisierung durch Benchmark Datasets evaluiert, die ground truth Bilder enthalten. Es ergibt aber keinen Sinn für unserer Pipeline diese zu nutzen, da wir nicht nur binarisieren, sondern auch andere Schritte durchführen.
\end{comment}

\begin{comment}
  Wir werden die performance (Laufzeit) der Pipeline anhand verschiedener Einstellungen testen. Bspw. unterschiediche Anzahl an Threads, verschieden große Bilder, verschiedene Kombinationen von Methoden.
\end{comment}


\section{Conclusions}
\label{sec:conclusions}


%%
%% The next two lines define the bibliography style to be used, and
%% the bibliography file.
\bibliographystyle{ACM-Reference-Format}
\bibliography{literature}


\end{document}
\endinput
%%
%% End of file `sample-sigconf.tex'.
