\documentclass[sigconf]{acmart}
\usepackage{booktabs}
\usepackage{algorithm}
\usepackage{url}


%% \BibTeX command to typeset BibTeX logo in the docs
\AtBeginDocument{%
  \providecommand\BibTeX{{%
    \normalfont B\kern-0.5em{\scshape i\kern-0.25em b}\kern-0.8em\TeX}}}

%% These commands are for a PROCEEDINGS abstract or paper.
\settopmatter{printacmref=true} % Removes citation information below abstract
%% Was false before, we need to check why its not working as intended
\renewcommand\footnotetextcopyrightpermission[1]{} % removes footnote with conference information in 

\acmConference[AEPRO 2026]{AEPRO 2026: Algorithm Engineering Projects}{March 1}{Jena, Germany}

% convert text to title case
% http://individed.com/code/to-title-case/

% that helps you to formulate your sentences
% https://www.deepl.com/translator

\begin{document}

%%
%% The "title" command has an optional parameter,
%% allowing the author to define a "short title" to be used in page headers.
%% TODO: think about good title
\title{Image Enhancement: A Lightweight, Easy-to-Use, and Efficient C++ Library}
\subtitle{\large Algorithm Engineering 2026 Project}

%%
%% The "author" command and its associated commands are used to define
%% the authors and their affiliations.

\author{Daniel Motz}
\affiliation{%
  \institution{Friedrich Schiller University}
  \city{Jena}
  \country{Germany}}
\email{daniel.motz@uni-jena.de}

\author{Leonard Teschner}
\affiliation{%
  \institution{Friedrich Schiller University}
  \city{Jena}
  \country{Germany}}
\email{leonard.teschner@uni-jena.de}

\author{Mher Mnatsakanyan}
\affiliation{%
  \institution{Friedrich Schiller University}
  \city{Jena}
  \country{Germany}}
\email{mher.mnatsakanyan@uni-jena.de}

%% The abstract is a short summary of the work to be presented in the article.
\begin{abstract}
  TBC
\end{abstract}

%%
%% Keywords. The author(s) should pick words that accurately describe
%% the work being presented. Separate the keywords with commas.
\keywords{task parallelism, WebAssembly, OpenMP, image processing, C++}

%%
%% This command processes the author and affiliation and title
%% information and builds the first part of the formatted document.
\maketitle

\section{Introduction}
\label{sec:intro}
\begin{comment}
  Possible Research Questions
  \begin{enumerate}
    \item Easily extensible Open Source Pipeline, ready to use out of the box with default options giving a perfectly good image, completely written in C++ and licensed under MIT-0 and CeCILL-C, which are both commercially available (compared to e.g. unpaper) and has no big dependencies (unlike unpaper, which uses ffmpeg) and only uses CImg, which is portable (uses OpenMP), simple, self-contained, and lightweight.
    \item Evaluating Performance Gains: How do systematic code optimizations improve the processing speed of a baseline image enhancement tool?
    \item Scaling Performance in Image Processing: Identifying and mitigating inefficiencies in an initial software implementation for OCR 
  \end{enumerate}
\end{comment}

\subsection{Background}
\label{sec:background}

The digitization of documents, especially images scanned or captured with smartphone cameras, is an important step in making information available. It also forms the basis for further processing steps such as optical character recognition (OCR). Existing tools and libraries for image enhancement and document preprocessing are powerful, but often too complex, too specialized, or have many dependencies. This makes it difficult to integrate them into small, user-defined programs and leads to the bloating of these programs. There is therefore a need for a lightweight, easy-to-use, and efficient solution that focuses on image enhancement while being flexible and modular enough to be used in a variety of applications.
As modern smartphone cameras capture images with ever-higher resolutions, the computing time required to process them also increases. Efficient implementation that utilizes the advantages of parallelization and other optimization techniques is therefore important for improving processing speed and ensuring user satisfaction.

\subsection{Related Work}
\label{sec:related-work}

\begin{comment}
Since the introduction of WASM in 2017 \cite{WASM_Haas_2017}, research has progressed in a variety of ways. For instance, significant investment has gone into the 'internal' WASM runtime, either to add features or to investigate its performance and energy efficiency . Another area of research has investigated the use of WASM in various application \cite{wasm_runtimes_survey_zhang_2025} areas, such as web applications, the Internet of Things, and high-performance computing \cite{wasm_runtimes_survey_zhang_2025}. 
Of particular mention is the work of Chadha et al., who examine the use of WASM in the HPC domain in \cite{Mpi_Wasm_Chadha_2023}. Since parallelisation is essential for HPC applications, they have developed a WASM embedder that enables the MPI library to be used in WASM applications.  
\end{comment}
\begin{comment}
  \begin{enumerate}
    \item easy to use
    \item out of the box good results
    \item low dependencies (only CImg (doesnt have other dependencies) and OpenMP )
    \item only image enhancement, no OCR
    \item MIT-0 or CeCILL-C license (commercially usable)
    \item modular (users can choose which steps to apply)
  \end{enumerate}

  Tools die Document Enhancement bereits können:
  - OpenCV -> viel zu groß und abhöngig von vielen anderen Bibliotheken, BSD Lizenz
  - Tesseract - Open Source OCR Engine, die einige Bildverbesserungsfunktionen bietet, aber hauptsächlich auf OCR ausgerichtet ist, nicht einfach erweiterbar, preprocessing sind nicht als allgemeine Enhnancement Methoden gedacht, Apache lizenz
  - Leptonica - C-Bibliothek für Bildverarbeitung, hauptsächlich auf OCR ausgerichtet, BCD Lizenz
  - ImageMagick - Umfassende Kommandozeilen-Tool und Bibliothek für Bildbearbeitung, aber nicht speziell auf die Verbesserung von gescannten oder fotografierten Textbildern ausgerichtet, viele Abhängigkeiten, nicht so einfach erweiterbar, Apache Lizenz
  - Unpaper - spezialisiert auf Nachbearbeitung von Scans von Büchern, viele Abhängigkeiten (ffmpeg etc), nicht so einfach erweiterbar und stark spezialisiert, GPL Lizenz
\end{comment} 

Image enhancement and document preprocessing are covered by a wide range of open-source tools and commercial software. The scope ranges from general purpose image processing libraries and OCR (optical character recognition) libraries to specialized tools for post-processing scanned documents.

\label{sec:general-purpose-libraries}
One of the most widely used libraries for general image processing is OpenCV \cite{opencv}. Described by Bradski in 2000 \cite{opencv_bradski_2000}, OpenCV offers a wide range of functions and algorithms for image processing, such as filters, transformations, segmentation, and feature detection, as well as machine and deep learning. OpenCV is written in C++ and has a modular, cross-platform architecture that allows for easy integration into various applications. However, OpenCV is a very complex library with many functions and dependencies, which leads to overhead when only a fraction of the functions are used. OpenCV is also licensed under the BSD license. 
Similar to OpenCV, ImageMagick \cite{imagemagick} is a comprehensive suite for image processing that is available as a command line tool and as a library. It offers functions such as format conversion, resizing, filtering, and color correction. However, ImageMagick is not specifically designed to enhance scanned or photographed images, but rather for general image processing. The wide range of functions and possibilities results in a large number of dependencies and a large code base. ImageMagick is licensed under the Apache License.

\label{sec:ocr-libraries}
The open source OCR engine Tesseract is often used for document analysis. Tesseract was described by Smith in 2007 \cite{Tesseract_OCR_Smith_2007} and has become one of the leading OCR engines. In addition to text recognition, Tesseract also offers some image enhancement features, but the focus is on OCR. The image enhancement features are typically integrated into the OCR workflow and are not available as general image enhancement features. Tesseract is licensed under the Apache License.
A similar library specializing in OCR is the C-based Leptonica library \cite{leptonica}. Leptonica offers image processing functions that are widely used in the backend of OCR engines such as Tesseract. It provides functions such as binarization, noise reduction, and morphological operations, which are important for preparing images for OCR recognition. Although Leptonica is relatively lightweight compared to the other libraries, it primarily serves as a low-level library and does not offer ready-to-use, configurable processing pipelines, for example. It is also written in C, which makes integration into C++ applications difficult. Leptonica is licensed under the BSD license.

\label{sec:specialized-post-processing}
A more specialized tool is Unpaper \cite{unpaper}, which focuses on post-processing scanned documents. It offers features such as deskewing, binarization, and noise reduction that have been developed specifically for this purpose. However, Unpaper is highly specialized and cannot be flexibly integrated into custom C++ applications. Due to its configuration, Unpaper has many dependencies, which makes it difficult to use. Unpaper is licensed under the GPL license, which restricts commercial use.

\label{sec:intro-optimization}
\begin{comment}
  CoPilots Vorschlag:
  In addition to the tools and libraries described above, there is a large body of research on code optimization and performance improvement in image processing applications. For example, various techniques have been proposed to optimize the performance of image processing algorithms, such as parallelization, vectorization, and algorithmic improvements. However, these optimizations are often implemented in the context of specific libraries or applications and are not always easily transferable to other contexts. Moreover, the performance gains from these optimizations can vary widely depending on the specific use case and implementation details.
\end{comment}

\subsection{Our Contribution}
\label{sec:our-contribution}

Unlike the existing tools and libraries described, our approach is designed to be minimalistic, extensible, and user-friendly. Instead of being a comprehensive image processing library or OCR engine, we focus on a lightweight, modular image enhancement pipeline written entirely in C++ with only one dependency (CImg \cite{cimg}), which is itself lightweight and portable. 
We systematically optimize and improve our intuitive baseline implementation and evaluate the performance gains through various experiments. 
Compared to OpenCV and ImageMagick, our work is characterized by its deliberate focus on image enhancement and compact architecture. Unlike Tesseract and Leptonica, our pipeline is not integrated into an OCR workflow and limited by low-level primitive functions, but offers a configurable standalone solution that is ready for immediate use. Our pipeline is not exclusively specialized in the post-processing of scanned documents, but is designed as a general image enhancement pipeline that can be used for a variety of applications.
In addition, our licensing model (MIT-0 or CeCILL-C) ensures its use in research and commercial applications. By combining minimal dependencies, a modular architecture, portability through C++ and OpenMP, and a commercially usable license, our pipeline fills a gap between large-scale, general-purpose image processing libraries and highly specialized tools for document enhancement.

\subsection{Outline}
\label{sec:outline}

In section \ref{sec:pipeline}, we introduce the pipeline that we will implement and optimize. In section \ref{sec:optimazation}, we describe the optimization techniques used. These are evaluated in section \ref{sec:experiments}, where we compare the performance of the pipeline under different settings to the baseline implementation. Finally, we summarize the results in section \ref{sec:conclusions} and provide an outlook on possible future work.

\section{The Pipeline}
\label{sec:pipeline}




\section{Pipeline Optimizations}
\label{sec:optimazation}

\section{Experiments}
\label{sec:experiments}

\begin{comment}
  In der Literatur wird die Binarisierung durch Benchmark Datasets evaluiert, die ground truth Bilder enthalten. Es ergibt aber keinen Sinn für unserer Pipeline diese zu nutzen, da wir nicht nur binarisieren, sondern auch andere Schritte durchführen.
\end{comment}

\begin{comment}
  Wir werden die performance (Laufzeit) der Pipeline anhand verschiedener Einstellungen testen. Bspw. unterschiediche Anzahl an Threads, verschieden große Bilder, verschiedene Kombinationen von Methoden.
\end{comment}


\section{Conclusions}
\label{sec:conclusions}


%%
%% The next two lines define the bibliography style to be used, and
%% the bibliography file.
\bibliographystyle{ACM-Reference-Format}
\bibliography{literature}


\end{document}
\endinput
%%
%% End of file `sample-sigconf.tex'.
