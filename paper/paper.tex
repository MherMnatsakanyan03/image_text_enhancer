\documentclass[sigconf]{acmart}
\usepackage{booktabs}


%% \BibTeX command to typeset BibTeX logo in the docs
\AtBeginDocument{%
  \providecommand\BibTeX{{%
    \normalfont B\kern-0.5em{\scshape i\kern-0.25em b}\kern-0.8em\TeX}}}

%% These commands are for a PROCEEDINGS abstract or paper.
\settopmatter{printacmref=false} % Removes citation information below abstract
\renewcommand\footnotetextcopyrightpermission[1]{} % removes footnote with conference information in 

\acmConference[AEPRO 2026]{AEPRO 2026: Algorithm Engineering Projects}{March 1}{Jena, Germany}

% convert text to title case
% http://individed.com/code/to-title-case/

% that helps you to formulate your sentences
% https://www.deepl.com/translator

\begin{document}

%%
%% The "title" command has an optional parameter,
%% allowing the author to define a "short title" to be used in page headers.
%% TODO: think about good title
\title[Image Text Enhancer]{Image Text Enhancer\\\large Algorithm Engineering 2026 Project Paper}

%%
%% The "author" command and its associated commands are used to define
%% the authors and their affiliations.

\author{Daniel Motz}
\affiliation{%
  \institution{Friedrich Schiller University Jena}
  \country{Germany}}
\email{daniel.motz@uni-jena.de}

\author{Leonard Teschner}
\affiliation{%
  \institution{Friedrich Schiller University Jena}
  \country{Germany}}
\email{leonard.teschnner@uni-jena.de}

\author{Mher Mnatsakanyan}
\affiliation{%
  \institution{Friedrich Schiller University Jena}
  \country{Germany}}
\email{mher.mnatsakanyan@uni-jena.de}

%% The abstract is a short summary of the work to be presented in the article.
\begin{abstract}

The five-finger pattern \cite{macgilchrist2014}:
\begin{enumerate}
\item \textbf{Topic and background:} What topic does the paper deal with? What is the point of departure for your research? Why are you studying this now?
\item \textbf{Focus:} What is your research question? What are you studying precisely?
\item \textbf{Method:} What did you do?
\item \textbf{Key findings:} What did you discover?
\item \textbf{Conclusions or implications:} What do these findings mean? What broader issues do they speak to?
\end{enumerate}


\end{abstract}

%%
%% Keywords. The author(s) should pick words that accurately describe
%% the work being presented. Separate the keywords with commas.
\keywords{noise reduction, background removal, image filter}


%%
%% This command processes the author and affiliation and title
%% information and builds the first part of the formatted document.
\maketitle

\let\thefootnote\relax\footnotetext{AEPRO 2026, March 1, Jena, Germany. Copyright \copyright 2026 for this paper by its authors. Use permitted under Creative Commons License Attribution 4.0 International (CC BY 4.0).}

\section{Introduction}
\label{sec:intro}

\subsection{Background}
\label{sec:background}
Motivation:
- Viele handschriftliche oder gedruckte Dokumente werden heutzutage mit dem Smartphone, oder im Zuge der Digitalisierung mit Druckern gescannt. Dabei entstehen häufig Bilder mit schlechter Qualität, die schwer lesbar, für Mensch und Maschine sind. Schlechte Qualität ist in diesem Kontext durch schlechte Beleuchtung, Schatten, Verzerrungen, Rauschen oder ungleichmäßigen Kontrast gekennzeichnet.
- Solche Bilder sind problematisch, wenn sie weiterverwendet werden sollen, beispielsweise um sie zu archivieren, zu drucken oder mittels OCR in maschinenlesbaren Text umzuwandeln.

\subsection{Related Work}
\label{sec:related-work}

drei Paper wurden uns empfohlen:\\
- Adaptive Thresholding Methods for Documents Image Binarization \cite{adaptive_threshilding_methods_bataineh_2011}: Das Bild wird in Fenster aufgeteilt. Fuer jedes Fenster wird ein Thresholdingvalue $T$ berechnet. In jedem Fenster wird die minimum $\sigma_{min}$ und maximum $\sigma_{max}$ Standardabweichungen aller Fenster verwendet, um adaptiv den Threshold $T$ zu bestimmen. Jeder Pixelwert $i(x, y)$ des Fensters wird anschliessend mit dem Threshold verglichen: Ist der Wert kleiner wird der Pixel schwarz und sonst weiss.\\
\begin{gather}
  T = m_w - \frac{m_W^2-\sigma_W}{(m_g+\sigma_W) \times (\sigma_{adaptive}+\sigma_W)}\\
  \sigma_{adaptive} = \frac{\sigma_W-\sigma_{min}}{\sigma_{max}-\sigma_{min}}\\
  I(x,y)= \begin{cases}
    black, & \text{if } i(x,y) < T\\
    white, & \text{otherwise}
  \end{cases}
\end{gather}
- Adaptive Thresholding using the Integral Image \cite{adaptive_thresholding_integral_image_bradley_roth_2007}\\
- Binarization of historical document images using the local maximum and minimum \cite{Binarization_historical_document_images_local_maximum_minimum_su_2010}
\subsection{Our Contributions}
\label{sec:our-contributions}
- kostenloses Tool/executable um die Qualität von gescannten Bildern zu verbessern, bevor sie beispielsweise versendet, gedruckt oder archiviert werden.
- Verbessert die Arbeit der Related Work, durch ... %%TODO: konkretisieren
- Kann beispielsweise für die Digitalisierung von Büchern, Dokumenten und handschriftlichen Akten oder Notizen genutzt werden. So wird eine gute, digitale lesbarkeit und wiedervenwendung ermöglicht.
-OpenMP und C++ Implementierung mit CMake Build System.

\subsection{Outline}
\label{sec:outline}
Dieses Paper ist wie folgt gegliedert: In Abschnitt \ref{sec:algorithm} beschreiben wir den entwickelten Algorithmus. In Abschnitt \ref{sec:experiments} zeigen wir die performance unseres Algorithmus anhand von Experimenten. Abschließend fassen wir in Abschnitt \ref{sec:conclusions} unsere Ergebnisse zusammen und geben einen Ausblick auf mögliche zukünftige Arbeiten.

\section{The Algorithm}
\label{sec:algorithm}



\section{Experiments}
\label{sec:experiments}

\section{Conclusions}
\label{sec:conclusions}

%%
%% The next two lines define the bibliography style to be used, and
%% the bibliography file.
\bibliographystyle{ACM-Reference-Format}
\bibliography{literature}


\end{document}
\endinput
%%
%% End of file `sample-sigconf.tex'.
